\documentclass{beamer}
\usepackage[T1]{fontenc}
\usepackage[utf8]{inputenc}
\usepackage{lmodern}
\usepackage[brazil]{babel}
\usepackage[labelformat=empty]{caption}
\usepackage{graphicx}
\usepackage{color}

\definecolor{beamer@blendedblue}{rgb}{0.8, 0, 0}
\definecolor{covered}{gray}{0.65}
\definecolor{filecolor}{rgb}{0, 0.3, 0.7}
\usetheme{Luebeck}
\title[Heartbleed Bug: CVE-2014-0160]{Heartbleed Bug: CVE-2014-0160}
\author{Carlos Eduardo Leão Elmadjian \and Renan Fichberg}
\date{Novembro 11, 2014}
\institute{Instituto de Matemática e Estatística da Universidade de São Paulo (IME-USP)}

\expandafter\def\expandafter\insertshorttitle\expandafter{%
\insertshorttitle\hfill%
\insertframenumber\,/\,\inserttotalframenumber}

\begin{document}

\begin{frame}
	\titlepage
\end{frame}

\begin{frame}
\begin{center}
	\includegraphics[scale=0.4]{heartbleed.png}
\end{center}
\end{frame}

\begin{frame}
	\frametitle{Conteúdo}
	\begin{itemize}
		\item A biblioteca OpenSSL
		\item Heartbleed Bug
		\item Remediando o problema
	\end{itemize}
\end{frame}

\begin{frame}
	\frametitle{A biblioteca OpenSSL}
	\begin{itemize}
		\item O que é?
		\item Qual é a sua importância?
	\end{itemize}
\end{frame}

\begin{frame}
	\frametitle{A biblioteca OpenSSL.}
	\framesubtitle{O que é?}
	\begin{itemize}
		\item Implementação open-source dos protocolos SSL e TLS
		\item Escrita em C
		\item Permite comunicação criptografada entre máquinas
		\item Possui \textit{wrappers} que permitem seu uso em várias linguagens
		\item Multi-plataforma 
	\end{itemize}
\end{frame}

\begin{frame}
	\frametitle{A biblioteca OpenSSL}
	\begin{itemize}
		\item \textcolor{covered}{O que é?}
		\item Qual é a sua importância?
	\end{itemize}
\end{frame}

%TODO: Rever as informações desta frame
\begin{frame}
	\frametitle{A biblioteca OpenSSL.}
	\framesubtitle{Qual é a sua importância?}
	\begin{itemize}
		\item Cerca de 2/3 dos servidores web utilizam OpenSSL
		\item Facebook, Twitter, Google, Yahoo, Dropbox, Instagram...
		\item Biblioteca padrão de criptografia no Linux e BSD
	\end{itemize}
\end{frame}

\begin{frame}
	\frametitle{A biblioteca OpenSSL}
	\begin{itemize}
		\item \textcolor{covered}{O que é?}
		\item \textcolor{covered}{Qual é a sua importância?}
	\end{itemize}
\end{frame}

\begin{frame}
	\frametitle{Conteúdo}
	\begin{itemize}
		\item \textcolor{covered}{A biblioteca OpenSSL}
		\item Heartbleed Bug
		\item Remediando o problema
	\end{itemize}
\end{frame}

\begin{frame}
	\frametitle{Heartbleed Bug}
	\begin{itemize}
		\item O que é?
		\item Por que temê-lo?
		\item Como funciona?
		\item Versões afetadas
	\end{itemize}
\end{frame}

\begin{frame}
	\frametitle{Heartbleed Bug}
	\framesubtitle{O que é?}
	\begin{itemize}
		\item Encontrado em Dezembro de 2011
		\item Uma vulnerabilidade na troca de heartbeats (RFC6520)
		\item Afeta tanto cliente quanto servidor
		\item Não é um problema com a criptografia
	\end{itemize}
\end{frame}

\begin{frame}
	\frametitle{Heartbleed Bug}
	\begin{itemize}
		\item \textcolor{covered}{O que é?}
		\item Por que temê-lo?
		\item Como funciona?
		\item Versões afetadas
	\end{itemize}
\end{frame}

\begin{frame}
	\frametitle{Heartbleed Bug}
	\framesubtitle{Por que temê-lo?}
	\begin{itemize}
		\item Permite ler 64KB da máquina atacada
		\item Não é preciso ter acesso privilegiado
		\item É possível roubar qualquer coisa (senhas, mensagens, certificados, endereços de e-mail...)
		\item O roubo não deixa vestígios
	\end{itemize}
\end{frame}

\begin{frame}
	\frametitle{Heartbleed Bug}
	\begin{itemize}
		\item \textcolor{covered}{O que é?}
		\item \textcolor{covered}{Por que temê-lo?}
		\item Como funciona?
		\item Versões afetadas
	\end{itemize}
\end{frame}

\begin{frame}
	\frametitle{Heartbleed Bug}
	\framesubtitle{Como funciona?}
	\begin{itemize}
		\item O bug se encontra na implementação do heartbeat do OpenSSL
		\item O heartbeat é usado para evitar a renegociação entre peers
		\item Linha problemática:\\
		buf = OPENSSL\_malloc(1 + 2 + payload + padding);
		\item Permite que o usuário aloque uma quantidade arbitrária de memória para um heartbeat
		\item A memória não é checada pelo destino e o um pacote de até 64KB é devolvido com memória coletada do servidor
		\item Na memória lida do servidor pode haver qualquer tipo de dado sensível
	\end{itemize}
\end{frame}

\begin{frame}
	\frametitle{Heartbleed Bug}
	\begin{itemize}
		\item \textcolor{covered}{O que é?}
		\item \textcolor{covered}{Por que temê-lo?}
		\item \textcolor{covered}{Como funciona?}
		\item Versões afetadas
	\end{itemize}
\end{frame}

\begin{frame}
	\frametitle{Heartbleed Bug}
	\framesubtitle{Versões afetadas}
	\begin{itemize}
		\item Versões vulneráveis do OpenSSL: da 1.0.1 a 1.0.1f
		\item São mais de 20 versões afetadas!
		\item Tarball \textbf{\textcolor{filecolor}{openssl-1.0.1.tar.gz}} lançado em Março de 2012
		\item Tarball \textbf{\textcolor{filecolor}{openssl-1.0.1f.tar.gz}} lançado em Janeiro de 2014
		\item Tarball \textbf{\textcolor{filecolor}{openssl-1.0.1g.tar.gz}} lançado em Abril de 2014, sem o bug
		\item Foram mais de 2 anos com o bug (e conseqüentemente de \textit{exploits})!
	\end{itemize}
\end{frame}

\begin{frame}
	\frametitle{Heartbleed Bug}
	\begin{itemize}
		\item \textcolor{covered}{O que é?}
		\item \textcolor{covered}{Por que temê-lo?}
		\item \textcolor{covered}{Como funciona?}
		\item \textcolor{covered}{Versões afetadas}
	\end{itemize}
\end{frame}

\begin{frame}
	\frametitle{Conteúdo}
	\begin{itemize}
		\item \textcolor{covered}{A biblioteca OpenSSL}
		\item \textcolor{covered}{Heartbleed Bug}
		\item Remediando o problema
	\end{itemize}
\end{frame}

\begin{frame}
	\frametitle{Remediando o problema}
	\begin{itemize}
		\item Opção 1: compilar com a tag:\\ \hspace{5 mm}-DOPENSSL\_NO\_HEARTBEATS
		\item Opção 2: usar a partir da versão \textbf{1.0.1g}
	\end{itemize}
\end{frame}

\begin{frame}
	\frametitle{Conteúdo}
	\begin{itemize}
		\item \textcolor{covered}{A biblioteca OpenSSL}
		\item \textcolor{covered}{Heartbleed Bug}
		\item \textcolor{covered}{Remediando o problema}
	\end{itemize}
\end{frame}

\begin{frame}
\begin{center}
	\includegraphics[scale=0.4]{heartcured.png}
	\begin{center} 
		Obrigado! :)
	\end{center}
\end{center}
\end{frame}

\begin{frame}
	\frametitle{Referências}
	\begin{itemize}
		\item \textit{http://heartbleed.com/}
		\item \textit{http://edition.cnn.com/2014/04/08/tech/web/heartbleed-openssl/}
		\item \textit{http://www.codenomicon.com/news/news/2014-05-20.shtml}
	\end{itemize}
\end{frame}

\end{document}